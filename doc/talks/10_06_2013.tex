%%%%%%%%%%%%%%%%%%%%%%%%%%%%%%%%%%%%%%%%%
% Beamer Presentation
% LaTeX Template
% Version 1.0 (10/11/12)
%
% This template has been downloaded from:
% http://www.LaTeXTemplates.com
%
% License:
% CC BY-NC-SA 3.0 (http://creativecommons.org/licenses/by-nc-sa/3.0/)
%
%%%%%%%%%%%%%%%%%%%%%%%%%%%%%%%%%%%%%%%%%

%----------------------------------------------------------------------------------------
%	PACKAGES AND THEMES
%----------------------------------------------------------------------------------------

\documentclass{beamer}

\mode<presentation> {

% The Beamer class comes with a number of default slide themes
% which change the colors and layouts of slides. Below this is a list
% of all the themes, uncomment each in turn to see what they look like.

%\usetheme{default}
%\usetheme{AnnArbor}
%\usetheme{Antibes}
%\usetheme{Bergen}
%\usetheme{Berkeley}
%\usetheme{Berlin}
%\usetheme{Boadilla}
%\usetheme{CambridgeUS}
%\usetheme{Copenhagen}
%\usetheme{Darmstadt}
%\usetheme{Dresden}
%\usetheme{Frankfurt}
%\usetheme{Goettingen}
%\usetheme{Hannover}
%\usetheme{Ilmenau}
%\usetheme{JuanLesPins}
%\usetheme{Luebeck}
\usetheme{Madrid}
%\usetheme{Malmoe}
%\usetheme{Marburg}
%\usetheme{Montpellier}
%\usetheme{PaloAlto}
%\usetheme{Pittsburgh}
%\usetheme{Rochester}
%\usetheme{Singapore}
%\usetheme{Szeged}
%\usetheme{Warsaw}

% As well as themes, the Beamer class has a number of color themes
% for any slide theme. Uncomment each of these in turn to see how it
% changes the colors of your current slide theme.

%\usecolortheme{albatross}
%\usecolortheme{beaver}
%\usecolortheme{beetle}
%\usecolortheme{crane}
%\usecolortheme{dolphin}
%\usecolortheme{dove}
%\usecolortheme{fly}
%\usecolortheme{lily}
%\usecolortheme{orchid}
%\usecolortheme{rose}
%\usecolortheme{seagull}
%\usecolortheme{seahorse}
%\usecolortheme{whale}
%\usecolortheme{wolverine}

%\setbeamertemplate{footline} % To remove the footer line in all slides uncomment this line
%\setbeamertemplate{footline}[page number] % To replace the footer line in all slides with a simple slide count uncomment this line

%\setbeamertemplate{navigation symbols}{} % To remove the navigation symbols from the bottom of all slides uncomment this line
}

\usepackage{graphicx} % Allows including images
\usepackage{booktabs} % Allows the use of \toprule, \midrule and \bottomrule in tables

%----------------------------------------------------------------------------------------
%	TITLE PAGE
%----------------------------------------------------------------------------------------

\title[Market simulator]{FiQuant Market Microstructure Simulator} % The short title appears at the bottom of every slide, the full title is only on the title page

\author{Anton Kolotaev} % Your name
\institute[ECP] % Your institution as it will appear on the bottom of every slide, may be shorthand to save space
{
Ecole Centrale de Paris \\ % Your institution for the title page
\medskip
\textit{anton.kolotaev@gmail.com} % Your email address
}
\date{\today} % Date, can be changed to a custom date

\begin{document}

\begin{frame}
\titlepage % Print the title page as the first slide
\end{frame}

\begin{frame}
\frametitle{Overview} % Table of contents slide, comment this block out to remove it
\tableofcontents % Throughout your presentation, if you choose to use \section{} and \subsection{} commands, these will automatically be printed on this slide as an overview of your presentation
\end{frame}

%----------------------------------------------------------------------------------------
%	PRESENTATION SLIDES
%----------------------------------------------------------------------------------------

%------------------------------------------------
\section{Installation} % Sections can be created in order to organize your presentation into discrete blocks, all sections and subsections are automatically printed in the table of contents as an overview of the talk
%------------------------------------------------
\begin{frame}
\frametitle{Requirements}
\begin{itemize}
\item OS supported: Linux, Mac OS X, Windows
\item Browsers supported: Chrome, Firefox, Safari, Opera
\item Python 2.7
\item Python packages can be installed using \texttt{pip} or \texttt{easyinstall}:
\begin{itemize}
\item \href{http://home.gna.org/veusz/}{Veusz} (for graph plotting)
\item \href{http://flask.pocoo.org}{Flask} (to run a Web-server)
\item \href{https://pypi.python.org/pypi/blist/}{Blist} (sorted collections used by ArbitrageTrader) 
\end{itemize}
\item Source code downloadable from \href{http://sourceforge.net/p/marketsimulator/svn/HEAD/tree/DevAnton/v3/}{SourceForge}
\end{itemize}

\end{frame}

\section{Simulator components}
\subsection{Scheduler} % A subsection can be created just before a set of slides with a common theme to further break down your presentation into chunks
\begin{frame}
\frametitle{Scheduler}
\begin{itemize}
  \item Main class for every discrete event simulation system.
  \item Maintains a set of actions to fulfill in future and launches them according their action times: from older ones to newer.
\end{itemize}
Interface:
\begin{itemize}
  \item Event scheduling:
  \begin{itemize}
    \item \texttt{schedule(actionTime, handler)}
    \item \texttt{scheduleAfter(dt, handler)}
  \end{itemize}
  \item Simulation control:
  \begin{itemize}
    \item \texttt{workTill(limitTime)}
    \item \texttt{advance(dt)}
    \item \texttt{reset()}
  \end{itemize}
\end{itemize}
\end{frame}

%------------------------------------------------
\subsection{Order books} % A subsection can be created just before a set of slides with a common theme to further break down your presentation into chunks
\begin{frame}
\frametitle{Order book}
\begin{itemize}
  \item Represents a single asset traded in some market (Same asset traded in different markets would be represented by different order books)
  \item Matches incoming orders
  \item Stores unfulfilled limit orders in two order queues (\texttt{Asks} for sell orders and \texttt{Bids} for buy orders)
  \item Corrects limit order price with respect to tick size
  \item Imposes order processing fee
  \item Supports queries about order book structure
  \item Notifies listeners about trades and price changes
\end{itemize}
\end{frame}

%------------------------------------------------
\begin{frame}
\frametitle{Order book for a remote trader}
\begin{itemize}
  \item Models a trader connected to a market by a communication channel with non-negligible latency
  \item Introduces delay in information propagation from a trader to an order book and vice versa (so a trader has outdated information about market and orders are sent to the market with a certain delay)
  \item Assures correct order of messages: older messages always come earlier than newer ones
\end{itemize}
\end{frame}

%------------------------------------------------
\subsection{Orders} 
\begin{frame}
\frametitle{Basic orders}
Orders supported internally by an order book:
\begin{itemize}
  \item \texttt{Market(side, volume)}
  \item \texttt{Limit(side, price, volume)}
  \item \texttt{Cancel(limitOrder)}
\end{itemize}
Limit and market orders notifies their listeners about all trades they take part in.
Factory functions are usually used in order to create orders.
\end{frame}

%------------------------------------------------
\begin{frame}
\frametitle{Meta orders}
Follow order interface from trader's perspective (so they can be used instead of basic orders) but behave like a sequence of base orders from an order book point of view.
\begin{itemize}
  \item \texttt{Iceberg(volumeLimit, orderToSplit)} splits \texttt{orderToSplit} to pieces with volume less than \texttt{volumeLimit} and sends them one by one to an order book ensuring that only one order at time is processed there
  \item \texttt{AlwaysBest(volume, limitOrderFactory)} creates a limit-like order with given volume and the most attractive price, sends it to an order book and if the order book best price changes, cancels it and resends with a better price
  \item \texttt{WithExpiry(lifetime, limitOrderFactory)} sends a limit-like order and after \texttt{lifetime} cancels it
  \item \texttt{LimitMarket(limitOrderFactory)} is like \texttt{WithExpiry} but with \texttt{lifetime} equal to 0 
\end{itemize}
\end{frame}

%------------------------------------------------
\subsection{Traders}
\begin{frame}
\frametitle{Traders}
Single asset traders
\begin{itemize}
  \item send orders to order books
  \item bookkeep their position and balance
  \item run a number of trading strategies
  \item notify listeners about trades done and orders sent
\end{itemize}
Single asset traders operate on a single or multiple markets.
If a trader holds a portfolio composed of multiple assets it should aggregate an array of single asset traders.
\end{frame}

%------------------------------------------------
\subsection{Strategies}
\begin{frame}[fragile]
\frametitle{Generic strategy}
\begin{verbatim}
class Generic(Strategy):
  def __init__(self, eventGen, sideFunc, ...):
    # ... storing constructor arguments
    event.subscribe(self.eventGen, self.wakeUp)

  def wakeUp(self):
    if not self.suspended:
      # determine side and parameters of an order to create
      side = self.sideFunc()
      if side <> None:
        volume = int(self.volumeFunc())
        if volume > 0:
          # create order given side and parameters
          order = self.orderFactory(side)(volume)
          # send order to the order book
          self.trader.send(order)
\end{verbatim}
\end{frame}

%------------------------------------------------

\begin{frame}
\frametitle{Bullet Points}
\begin{itemize}
\item Lorem ipsum dolor sit amet, consectetur adipiscing elit
\item Aliquam blandit faucibus nisi, sit amet dapibus enim tempus eu
\item Nulla commodo, erat quis gravida posuere, elit lacus lobortis est, quis porttitor odio mauris at libero
\item Nam cursus est eget velit posuere pellentesque
\item Vestibulum faucibus velit a augue condimentum quis convallis nulla gravida
\end{itemize}
\end{frame}

%------------------------------------------------

\begin{frame}
\frametitle{Blocks of Highlighted Text}
\begin{block}{Block 1}
Lorem ipsum dolor sit amet, consectetur adipiscing elit. Integer lectus nisl, ultricies in feugiat rutrum, porttitor sit amet augue. Aliquam ut tortor mauris. Sed volutpat ante purus, quis accumsan dolor.
\end{block}

\begin{block}{Block 2}
Pellentesque sed tellus purus. Class aptent taciti sociosqu ad litora torquent per conubia nostra, per inceptos himenaeos. Vestibulum quis magna at risus dictum tempor eu vitae velit.
\end{block}

\begin{block}{Block 3}
Suspendisse tincidunt sagittis gravida. Curabitur condimentum, enim sed venenatis rutrum, ipsum neque consectetur orci, sed blandit justo nisi ac lacus.
\end{block}
\end{frame}

%------------------------------------------------

\begin{frame}
\frametitle{Multiple Columns}
\begin{columns}[c] % The "c" option specifies centered vertical alignment while the "t" option is used for top vertical alignment

\column{.45\textwidth} % Left column and width
\textbf{Heading}
\begin{enumerate}
\item Statement
\item Explanation
\item Example
\end{enumerate}

\column{.5\textwidth} % Right column and width
Lorem ipsum dolor sit amet, consectetur adipiscing elit. Integer lectus nisl, ultricies in feugiat rutrum, porttitor sit amet augue. Aliquam ut tortor mauris. Sed volutpat ante purus, quis accumsan dolor.

\end{columns}
\end{frame}

%------------------------------------------------
\section{Second Section}
%------------------------------------------------

\begin{frame}
\frametitle{Table}
\begin{table}
\begin{tabular}{l l l}
\toprule
\textbf{Treatments} & \textbf{Response 1} & \textbf{Response 2}\\
\midrule
Treatment 1 & 0.0003262 & 0.562 \\
Treatment 2 & 0.0015681 & 0.910 \\
Treatment 3 & 0.0009271 & 0.296 \\
\bottomrule
\end{tabular}
\caption{Table caption}
\end{table}
\end{frame}

%------------------------------------------------

\begin{frame}
\frametitle{Theorem}
\begin{theorem}[Mass--energy equivalence]
$E = mc^2$
\end{theorem}
\end{frame}

%------------------------------------------------

\begin{frame}[fragile] % Need to use the fragile option when verbatim is used in the slide
\frametitle{Verbatim}
\begin{example}[Theorem Slide Code]
\begin{verbatim}
\begin{frame}
\frametitle{Theorem}
\begin{theorem}[Mass--energy equivalence]
$E = mc^2$
\end{theorem}
\end{frame}\end{verbatim}
\end{example}
\end{frame}

%------------------------------------------------

\begin{frame}
\frametitle{Figure}
Uncomment the code on this slide to include your own image from the same directory as the template .TeX file.
%\begin{figure}
%\includegraphics[width=0.8\linewidth]{test}
%\end{figure}
\end{frame}

%------------------------------------------------

\begin{frame}[fragile] % Need to use the fragile option when verbatim is used in the slide
\frametitle{Citation}
An example of the \verb|\cite| command to cite within the presentation:\\~

This statement requires citation \cite{p1}.
\end{frame}

%------------------------------------------------

\begin{frame}
\frametitle{References}
\footnotesize{
\begin{thebibliography}{99} % Beamer does not support BibTeX so references must be inserted manually as below
\bibitem[Smith, 2012]{p1} John Smith (2012)
\newblock Title of the publication
\newblock \emph{Journal Name} 12(3), 45 -- 678.
\end{thebibliography}
}
\end{frame}

%------------------------------------------------

\begin{frame}
\Huge{\centerline{The End}}
\end{frame}

%----------------------------------------------------------------------------------------

\end{document} 